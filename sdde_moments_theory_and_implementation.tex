
\documentclass[11pt]{article}
\usepackage[a4paper,margin=1in]{geometry}
\usepackage{amsmath,amssymb,mathtools,bm}
\usepackage{physics}
\usepackage{microtype}
% \usepackage{hyperref}
\usepackage{upquote}
\usepackage{listings}
\usepackage{xcolor}
\lstdefinelanguage{Julia}%
  {morekeywords={abstract,break,case,catch,const,continue,do,else,elseif,end,%
      export,false,for,function,immutable,import,importall,if,in,macro,module,%
      quote,return,switch,true,try,type,typealias,using,while,where,struct,mutable,primitive},
   sensitive=true,%
   morecomment=[l]\#,%
   morecomment=[n]{\#=}{=\#},%
   morestring=[b]",%
}
\lstset{
  language=Julia,
  basicstyle=\ttfamily\small,
  keywordstyle=\bfseries\color{blue!60!black},
  commentstyle=\itshape\color{green!40!black},
  stringstyle=\color{orange!60!black},
  showstringspaces=false,
  columns=fullflexible,
  frame=single,
  breaklines=true,
  postbreak=\mbox{\textcolor{red}{$\hookrightarrow$}\space},
  tabsize=2
}

\title{Linear Affine SDDEs: Moment Equations, Method of Steps, and a Julia Implementation}
\author{}
\date{\today}

\begin{document}
\maketitle

\begin{abstract}
We derive and implement deterministic differential--delay equations (DDEs) for the first and second moments of a linear, affine stochastic delay differential equation (SDDE) with affine multiplicative noise. The result is a closed system for the mean, the second (raw) moment, and a ladder of cross-moments at integer multiples of the delay. We then show how to integrate these DDEs in \texttt{Julia} using the \texttt{DifferentialEquations.jl} ecosystem, covering practical details such as state layout, absolute-time history access, time-varying coefficients, and validation in a scalar test case with known behavior.
\end{abstract}

\section{Model}
Consider the $n$-dimensional SDDE
\begin{equation}\label{eq:sdde}
  \dd x(t) = \big(Ax(t) + B x(t-\tau) + c\big)\,\dd t
   \;+\; \big(\alpha x(t) + \beta x(t-\tau) + \gamma\big)\,\dd W_t,
\end{equation}
where $A,B,\alpha,\beta\in\mathbb{R}^{n\times n}$, $c,\gamma\in\mathbb{R}^n$, $\tau>0$, and $W_t$ is a scalar Wiener process.\footnote{For $m$-dimensional $W_t$ with covariance $Q$, replace $LL^\top$ by $LQL^\top$ throughout.} We assume an initial segment (history) $x(s)=\phi(s)$ for $s\in[-\tau,0]$ (deterministic unless stated otherwise).

Define the mean $\mu(t)=\mathbb{E}[x(t)]$, the raw second moment $M(t)=\mathbb{E}[x(t)x(t)^\top]$, and for $k\ge1$ the cross-moments
\begin{equation}
  S_k(t) \;=\; \mathbb{E}\!\left[x(t)\,x\big(t-k\tau\big)^\top\right].
\end{equation}
Denote $N(t)=M(t-\tau)$ and $C(t)=S_1(t)$.

\section{Moment equations}
Taking expectations of \eqref{eq:sdde} gives the closed mean DDE
\begin{equation}\label{eq:mean}
  \dot\mu(t) = A\,\mu(t) + B\,\mu(t-\tau) + c.
\end{equation}
For the second moment, apply It\^o's product rule to $x(t)x(t)^\top$ and note that the quadratic variation contributes $L(t)L(t)^\top$ with
$L(t)=\alpha x(t)+\beta x(t-\tau)+\gamma$:
\begin{align}\label{eq:Mdot}
  \dot M(t) &= A M(t) + M(t)A^\top + B M(t-\tau) + C(t) B^\top + c\,\mu(t)^\top + \mu(t)\,c^\top
  \nonumber\\
  &\quad + \mathbb{E}\!\left[L(t)L(t)^\top\right],
\end{align}
and a standard expansion yields
\begin{align}\label{eq:ELLT}
\mathbb{E}\!\left[L L^\top\right] &=
  \alpha M\alpha^\top + \beta N\beta^\top + \alpha C\beta^\top + \beta C^\top\alpha^\top
  \nonumber\\
&\quad + \alpha\mu\,\gamma^\top + \gamma\mu^\top\alpha^\top
      + \beta\mu_\tau\,\gamma^\top + \gamma\mu_\tau^\top\beta^\top + \gamma\gamma^\top,
\end{align}
where $\mu_\tau=\mu(t-\tau)$ and $N=M(t-\tau)$.

For cross-moments there is no It\^o correction because $x(t-k\tau)$ has no differential at time $t$.
For $C(t)=S_1(t)$ one gets
\begin{equation}\label{eq:Cd}
  \dot C(t) = A C(t) + C(t)A^\top + B M(t-\tau) + S_2(t) B^\top + c\,\mu(t-\tau)^\top + \mu(t)\,c^\top.
\end{equation}
For $D(t)=S_2(t)$ (and similarly onward),
\begin{equation}\label{eq:Dd}
  \dot S_2(t) = A S_2(t) + S_2(t)A^\top + B C(t-\tau) + S_3(t) B^\top + c\,\mu(t-2\tau)^\top + \mu(t)\,c^\top.
\end{equation}
\color{red}
Azért itt lehetnek buktatók, ha időfüggőek az együtthatók, mert lehet, hogy pl $c(t-\tau)$ meg ilyenek jelennek meg, meg kell nézni.
\color{black}
In general, for $k\ge1$,
\begin{equation}\label{eq:Sk}
  \boxed{\;\dot S_k(t) = A S_k + S_k A^\top + B\,S_{k-1}(t-\tau) + S_{k+1} B^\top + c\,\mu(t-k\tau)^\top + \mu(t)\,c^\top,\;}
\end{equation}
with the convention $S_0(t)=M(t)$ and $S_{-1}(t)\equiv M(t-\tau)$ when $k=1$.

\paragraph{Covariance form.}
Let $P(t)=M(t)-\mu(t)\mu(t)^\top$ and $R_k(t)=S_k(t)-\mu(t)\mu(t-k\tau)^\top$. One can rewrite \eqref{eq:Mdot,eq:Sk} for $(P,R_k)$, but for implementation it is simpler to evolve raw moments and form covariances a posteriori.

\paragraph{Steady state (delay Lyapunov).}
If the system is mean-square stable, cross-covariances $S(\theta)=\lim_{t\to\infty}\mathbb{E}[x(t)x(t-\theta)^\top]$ satisfy $\dv*{S}{\theta}=A S$ for $\theta\in(0,\tau)$ with $S(0)=P$, yielding $S(\tau)=e^{A\tau}P$ and the algebraic \emph{delay Lyapunov} equation
\begin{equation}
  A P + P A^\top + B S(\tau)^\top + S(\tau) B^\top + \Gamma = 0,\qquad \Gamma=\mathbb{E}[L L^\top]_{\text{ss}}.
\end{equation}

\section{Method of steps and closure on a finite horizon}
On $[0,\tau]$, all delayed quantities (e.g.\ $M(t-\tau)$, $\mu(t-\tau)$, $S_k(t-\tau)$) depend only on the history and are therefore known functions. Hence \eqref{eq:mean,eq:Mdot,eq:Sk} form a closed linear ODE in time. On each subsequent interval $[m\tau,(m+1)\tau]$, the delayed terms are known from the previously computed solution, so again we solve a closed ODE. For a finite horizon $[0,T]$, choosing $K=\lfloor T/\tau\rfloor$ ensures $t-(K+1)\tau<0$ for all $t\in[0,T]$, so the highest cross-moment that appears, $S_{K+1}(t)$, multiplies the \emph{history} only and is known:
\begin{equation}
  S_{K+1}(t) \;=\; \mu(t)\,\phi\!\big(t-(K+1)\tau\big)^\top \quad\text{(deterministic history)}.
\end{equation}

\section{Implementation in Julia}
We integrate the moment DDEs using \texttt{DifferentialEquations.jl}. The state is
\begin{equation*}
  y(t) = \big[\ \mu(t);\ \mathrm{vec}(M(t));\ \mathrm{vec}(S_1(t));\ \dots;\ \mathrm{vec}(S_K(t))\ \big].
\end{equation*}
We define a layout to pack/unpack $\mu$, $M$, and $\{S_k\}$. The RHS evaluates the coefficient matrices/vectors (possibly time-varying) at the current $t$, uses the solver-supplied \emph{absolute-time} history accessor $h(p,s)$ to fetch past states at $s=t-\tau$ and $s=t-k\tau$, and forms the derivatives from \eqref{eq:mean,eq:Mdot,eq:Sk}.

\paragraph{Absolute-time history.}
In the \texttt{DDEProblem} API, the history accessor is \emph{absolute time}: calling \texttt{h(p, s)} returns the state at time $s$ (for $s\le0$ the user-provided history; for $s>0$ the previously integrated/interpolated solution). This is why we use \texttt{h(p, t - \textit{lag})} rather than offsets.

\subsection*{Core code (excerpt)}
\begin{lstlisting}
# coefficients can be constants or functions of time
mat_at(M, t) = M isa Function ? M(t) : M
vec_at(v, t) = v isa Function ? v(t) : v

function E_LLᵀ(αt, βt, γt, μ, μτ, M, C, N)
    αt*M*αt' + βt*N*βt' + αt*C*βt' + βt*C'*αt' +
    αt*μ*γt' + γt*μ'*αt' + βt*μτ*γt' + γt*μτ'*βt' + γt*γt'
end

function mom_rhs!(dy, y, h, p, t)
    A = mat_at(p.A,t); B = mat_at(p.B,t)
    α = mat_at(p.α,t); β = mat_at(p.β,t)
    c = vec_at(p.c,t); γ = vec_at(p.γ,t)
    μ, M, S = unpack_state(y, p.layout)

    # past states at absolute times
    μτ, Mτ, Sτ = unpack_state(h(p, t - p.τ), p.layout)
    μedge, _, _ = unpack_state(h(p, t - (p.K+1)*p.τ), p.layout)
    S_Kp1 = μ * μedge'

    C = (p.K>=1) ? S[1] : zeros(p.layout.n, p.layout.n)

    dμ = A*μ + B*μτ + c
    dM = A*M + M*A' + B*Mτ + C*B' + c*μ' + μ*c' +
         E_LLᵀ(α, β, γ, μ, μτ, M, C, Mτ)

    for k in 1:p.K
        Sk = S[k]
        Skm1_delay = (k==1) ? Mτ : Sτ[k-1]
        Skp1 = (k<p.K) ? S[k+1] : S_Kp1
        μkm, _, _ = unpack_state(h(p, t - k*p.τ), p.layout)
        dSk = A*Sk + Sk*A' + B*Skm1_delay + Skp1*B' + c*μkm' + μ*c'
        dy[p.layout.idx_S[k]] .= vec(dSk)
    end

    dy[p.layout.idx_μ] .= dμ
    dy[p.layout.idx_M] .= vec(dM)
end
\end{lstlisting}

The full implementation (packing, history builder, driver, etc.) matches the code you now use. Coefficients may be either constants or functions $t\mapsto$ matrix/vector, and the history is supplied by a user function $\phi(t)$ returning the deterministic initial state when $t\le0$.

\section{Validation on a scalar test}
Take $n=1$ with
$A=-6$, $B=0$, $\alpha=0$, $\beta=2$, $\gamma=1$, $\tau=1$, and deterministic history $\phi\equiv 3$.
On $[0,1]$,
\begin{align*}
  \dot M &= -12 M + \underbrace{\big(4M(t-\tau)+4\mu(t-\tau)+1\big)}_{\mathbb{E}[(\beta x_{-\tau}+\gamma)^2]}
          = -12M + 49,\\
  \mu(t) &= 3 e^{-6t}, \qquad M(0)=9.
\end{align*}
Therefore the variance on the first step is
\begin{equation*}
  \operatorname{Var}(t)=M(t)-\mu(t)^2=\frac{49}{12}\big(1-e^{-12t}\big),\qquad 0\le t\le 1,
\end{equation*}
which rises from $0$ to $\approx 4.083$ by $t=1$, and then begins to bend as the delayed inputs $M(t-1)$ and $\mu(t-1)$ decay. The numerical integration matches this shape when the absolute-time history access is used consistently in the RHS.

\section{Extensions}
\paragraph{Multi-dimensional noise.}
If $W$ is $m$-dimensional with covariance $Q$ (possibly time-varying), the only change is
\begin{equation*}
  \mathbb{E}[L L^\top]\ \longrightarrow\ \mathbb{E}[L Q L^\top]
\end{equation*}
in \eqref{eq:Mdot}, i.e.\ insert $Q$ between the left/right factors in each bilinear term.

\paragraph{Random history.}
If the history is random with known first/second moments and independent of future noise, build the history state $y(t)$ for $t\le0$ from those moments instead of $\phi\phi^\top$. In that case $S_{K+1}(t)$ should be computed via the history accessor to use the correct cross-moment with the random history.

\paragraph{Covariances directly.}
One can evolve $P$ and $R_k$ directly; evolving raw moments tends to be simpler and numerically robust, with covariances formed as $P=M-\mu\mu^\top$ and $R_k=S_k-\mu\,\mu(\cdot)^\top$ for diagnostics.

% \section{Common pitfalls}
% \begin{itemize}
%   \item \textbf{Absolute vs.\ offset time in history access.} In \texttt{DifferentialEquations.jl}, call \texttt{h(p, t-\tau)} to fetch $t-\tau$, not \texttt{h(p, -\tau)}.
%   \item \textbf{Forgetting constant--mean cross terms.} The $c\,\mu(\cdot)^\top$ and $\mu\,c^\top$ pieces appear in all cross-moment equations.
%   \item \textbf{Using deterministic history assumptions inadvertently.} If you later switch to random history, make sure to adapt $y(t)$ for $t\le0$ and compute $S_{K+1}(t)$ via \texttt{h}.
% \end{itemize}

\section{Usage outline}
\begin{lstlisting}
# Define A,B,α,β as constants or functions t->Matrix; c,γ as vectors or t->Vector
n = 2; A(t) = -0.8I(n); B(t) = 0.1I(n); α(t) = 0.3I(n); β(t) = 0.1I(n)
c(t) = zeros(n); γ(t) = [0.2, 0.0]
τ = 0.7; T = 5.0
φ(t) = [1.0, 0.0]             # deterministic history for t ≤ 0

sol, L = solve_moments(A,B,c,α,β,γ; τ=τ, T=T, φ=φ, saveat=0:0.05:T)
μt, Mt, St = get_moments_at(sol, L, 3.0)   # read moments at t=3.0
\end{lstlisting}

\section{Conclusion}
The linear--affine SDDE \eqref{eq:sdde} admits a closed deterministic DDE system for the mean, second moment, and a finite ladder of cross-moments on any finite horizon. Implemented with the method of steps and absolute-time history access, the approach is efficient and stable in practice, and readily accommodates time-varying coefficients and multi-dimensional noise.
\end{document}

